% Options for packages loaded elsewhere
\PassOptionsToPackage{unicode}{hyperref}
\PassOptionsToPackage{hyphens}{url}
%
\documentclass[
]{book}
\usepackage{lmodern}
\usepackage{amssymb,amsmath}
\usepackage{ifxetex,ifluatex}
\ifnum 0\ifxetex 1\fi\ifluatex 1\fi=0 % if pdftex
  \usepackage[T1]{fontenc}
  \usepackage[utf8]{inputenc}
  \usepackage{textcomp} % provide euro and other symbols
\else % if luatex or xetex
  \usepackage{unicode-math}
  \defaultfontfeatures{Scale=MatchLowercase}
  \defaultfontfeatures[\rmfamily]{Ligatures=TeX,Scale=1}
\fi
% Use upquote if available, for straight quotes in verbatim environments
\IfFileExists{upquote.sty}{\usepackage{upquote}}{}
\IfFileExists{microtype.sty}{% use microtype if available
  \usepackage[]{microtype}
  \UseMicrotypeSet[protrusion]{basicmath} % disable protrusion for tt fonts
}{}
\makeatletter
\@ifundefined{KOMAClassName}{% if non-KOMA class
  \IfFileExists{parskip.sty}{%
    \usepackage{parskip}
  }{% else
    \setlength{\parindent}{0pt}
    \setlength{\parskip}{6pt plus 2pt minus 1pt}}
}{% if KOMA class
  \KOMAoptions{parskip=half}}
\makeatother
\usepackage{xcolor}
\IfFileExists{xurl.sty}{\usepackage{xurl}}{} % add URL line breaks if available
\IfFileExists{bookmark.sty}{\usepackage{bookmark}}{\usepackage{hyperref}}
\hypersetup{
  pdftitle={Mixed Methods},
  pdfauthor={Ken Horvath},
  hidelinks,
  pdfcreator={LaTeX via pandoc}}
\urlstyle{same} % disable monospaced font for URLs
\usepackage{longtable,booktabs}
% Correct order of tables after \paragraph or \subparagraph
\usepackage{etoolbox}
\makeatletter
\patchcmd\longtable{\par}{\if@noskipsec\mbox{}\fi\par}{}{}
\makeatother
% Allow footnotes in longtable head/foot
\IfFileExists{footnotehyper.sty}{\usepackage{footnotehyper}}{\usepackage{footnote}}
\makesavenoteenv{longtable}
\usepackage{graphicx}
\makeatletter
\def\maxwidth{\ifdim\Gin@nat@width>\linewidth\linewidth\else\Gin@nat@width\fi}
\def\maxheight{\ifdim\Gin@nat@height>\textheight\textheight\else\Gin@nat@height\fi}
\makeatother
% Scale images if necessary, so that they will not overflow the page
% margins by default, and it is still possible to overwrite the defaults
% using explicit options in \includegraphics[width, height, ...]{}
\setkeys{Gin}{width=\maxwidth,height=\maxheight,keepaspectratio}
% Set default figure placement to htbp
\makeatletter
\def\fps@figure{htbp}
\makeatother
\setlength{\emergencystretch}{3em} % prevent overfull lines
\providecommand{\tightlist}{%
  \setlength{\itemsep}{0pt}\setlength{\parskip}{0pt}}
\setcounter{secnumdepth}{5}
\usepackage{booktabs}
\usepackage{fancyhdr}
\pagestyle{fancy}
\fancyhead[RE,LO]{\thepage}
\fancyhead[LE,RO]{Mixed Methods -- multimethodisch forschen}
\fancyfoot[CE]{}
\fancyfoot[CO]{}
\usepackage[ngerman]{babel}
\usepackage{babelbib}
\usepackage[]{natbib}
\bibliographystyle{apalike}

\title{Mixed Methods}
\author{Ken Horvath}
\date{}

\begin{document}
\maketitle

{
\setcounter{tocdepth}{1}
\tableofcontents
}
\hypertarget{multimethodisch-forschen-skriptum}{%
\chapter*{Multimethodisch forschen -- Skriptum}\label{multimethodisch-forschen-skriptum}}
\addcontentsline{toc}{chapter}{Multimethodisch forschen -- Skriptum}

``Mixed Methods'' haben sich in den letzten Jahren als eigenständiges Feld in der sozialwissenschaftlichen Methodenlandschaft etabliert. Hinter dem verlockend schlichten Label verbirgt sich ein heterogenes und dynamisches Gefüge an methodologischen Perspektiven und Positionen, das für die aktuelle Forschungslandschaft in vielen Hinsichten bedeutsam ist.

Vor diesem Hintergrund führen die folgenden Seiten in die zentrale Aspekte der Mixed-Methods-Forschung ein. Sie sind als digitale Seminarbegleitung gedacht und wollen (1) einen knappen, aber umfassenden Einblick in Logiken und Techniken multimethodischer Forschung geben und (2) auf einige Punkte aufmerksam machen, die für ein richtiges Verständnis der Thematik grundlegend sind, häufig aber nur unter der Oberfläche abgehandelt werden.

\hypertarget{intro}{%
\chapter{Einführung und Überblick}\label{intro}}

\hypertarget{wozu-mixed-methods}{%
\section{Wozu Mixed Methods?}\label{wozu-mixed-methods}}

In einer ersten, einfachen Definition werden mit Mixed Methods empirische Forschungsprojekte bezeichnet, in denen qualitative und quantitative Methoden kombiniert werden. Diese erste Annäherung ist hilfreich, weil sie eine zentrale Herausforderung betont: Verschiedene Methoden in einem Projekt oder Forschungskontext zusammenzuführen, die historisch häufig in aktiver Abgrenzung zueinander entwickelt wurden.

Entscheidend ist allerdings nicht so sehr der Umstand, dass qualitative und quantitative Methoden verbunden werden, sondern wie genau diese Verbindungen hergestellt, begründet und reflektiert werden. \emph{Wie} man also Methoden aufeinander wie auch auf den Forschungsgegenstand und die Zielsetzungen eines Projekts abstimmt, ist die entscheidende Frage.

Im Zuge dieser \emph{begründeten und reflektierten Kombination von Methoden} kommen zwangsläufig Fragen auf, die nicht nur für Mixed Methods wichtig sind, sondern von allgemeinem methodologischen Interesse sind:

\begin{itemize}
\item
  Das entscheidende Anliegen von Mixed Methods wird meist darin gesehen, dem \textbf{Forschungsvorhaben} angemessene Forschungsprozesse zu gestalten. Doch was bedeutet das genau und wie stellt man ``Angemessenheit'' von Methoden her?
\item
  Im Vergleich zum nach wie vor starken ``Werkzeugkasten-Denken'' in anderen Bereichen der Methodenlehre und -entwicklung rückt im Feld von Mixed Methods der Fokus auf \textbf{Designfragen} -- und damit auch auf \textbf{grundlegende Logiken} und \textbf{wissenschaftstheoretische Grundlagen} empirischer Forschung.
\item
  Die Orientierung auf Angemessenheit und Designfragen spiegelt sich in der expliziten Anerkennung der \textbf{Rolle von Forschungsfragen} wider. In kaum in einem anderen Bereich der sozialwissenschaftlichen Methodenlandschaft wird diesem Aspekt vergleichbar viel Aufmerksamkeit geschenkt.
\end{itemize}

Diese und andere Eigenheiten verweisen auf sozialwissenschaftliche Grundkompetenzen, für die in der traditionellen Methodenlehre in der Regel relativ wenig Zeit bleibt. Der Zweck der Beschäftigung mit Mixed Methods liegt daher nicht zuletzt darin, diese oft vernachlässigten \textbf{Kompetenzen zu trainieren}, selbst wenn aktuell an keinem eigenen qualitativ-plus-quantitativen Forschungsprojekt gebastelt wird. Denn erstens tragen diese Fertigkeiten auch in ``rein qualitativen'' und ``rein quantitativen'' Projekten zur bewussten Gestaltung von kohärenten und produktiven Forschungsprozessen bei. Zweitens geht es in den Sozialwissenschaften auch darum, mit den Forschungsarbeiten anderer umgehen zu können. Mixed Methods stehen in diesem Sinn auch für die Aneignung einer \textbf{Rezeptionsfertigkeit}: verstehen, was andere Forschende tun, und sich aktiv und kompetent auf ihre Ergebnisse beziehen können.

Dieser Logik folgend gliedert sich dieses Online-Buch grob in zwei Teile. Die ersten Abschnitte widmen sich grundlegenden Aspekten empirischer Forschung, die für ein Verständnis von Mixed Methods essenziell sind: die Unterscheidung von qualitativen und quantitativen Methoden, das Verhältnis von Design- und Methodenfragen, die Rolle von Forschungsfragen und die Bestimmung von Gegenständen und Zielen eines Forschungsprojekts. Die anschließenden Abschnitte wenden sich dann der Mixed-Methods-Forschung in ihrer aktuell vorherrschenden Form zu.

\hypertarget{inhalte-im-uxfcberblick}{%
\section{Inhalte im Überblick}\label{inhalte-im-uxfcberblick}}

Die Kapitel dieses Skriptums tasten sich von eher abstrakten Grundlagenfragen schrittweise zu konkreten Forschungsszenarien vor. Die folgenden Themen werden behandelt:

\begin{enumerate}
\def\labelenumi{\arabic{enumi}.}
\item
  \protect\hyperlink{designsmethoden}{\textbf{Designs und Methoden -- zwei Blickwinkel auf empirische Forschung}}: Für die Beschäftigung mit Mixed Methods ist die Differenzierung von Design- und Methodenfragen zentral. Mixed-Methods-Projekte sind häufig nicht nur multimethodisch, sondern kombinieren auch Facetten der drei sozialwissenschaftlichen Grunddesigns (Experiment, Erhebung, Fallstudie). Wer Methoden angemessen und bewusst verbinden will, muss auf beiden diesen Ebenen über Forschung nachdenken und kommunizieren.
\item
  \textbf{Mixed Methods und die Unterscheidung von qualitativen und quantitativen Methoden}: Die Unterscheidung von qualitativer und quantitativer Forschung ist für Mixed Methods von zentraler Bedeutung. Diese Unterscheidung ist einerseits irreführend, anderereits aber auch erhellend und in vielen Hinsichten angemessen. Ob sie sinnvoll ist oder nicht, ist letzten Endes eine Frage des Blickwinkels bzw. der Tiefenschärfe.
\item
  \protect\hyperlink{gegenstaendeziele}{\textbf{Gegenstände und Ziele als zentrale Bezugspunkte}}: Eine Antwort darauf, wie Forschungsfragen die Methodenwahl bestimmen können, lautet: Weil Forschungsfragen den Gegenstand und die (Generalisierungs-)Ziele eines Projekts definieren. Es ist daher entscheidend, diese beiden Aspekte eines Forschungsvorhabens klar zu fassen. Zur Zieldefinition gehört auch, den Grund für die Kombination qualitativer und quantitativer Methoden beim Namen zu nennen (kombiniern, um zu validieren, zu illustrieren, zu vertiefen, zu ergänzen \ldots).
\item
  \protect\hyperlink{forschungsfragen}{\textbf{Forschungsfragen und ihre Funktionen für die empirische Forschung}}: Das zentrale Mantra der Mixed-Methods-Forschung besagt, dass die Forschungsfrage die Methodenwahl bestimmen soll (und nicht umgekehrt Methodenvorlieben Fragen vorgeben sollen). Aber wie genau ergeben sich \emph{Methoden} aus einer \emph{Frage}? Was macht eine (gute) Forschungsfrage eigentlich genau aus? Wieso sind Forschungsfragen so wichtig? Und wie findet man sie?
\item
  \textbf{Mixed Methods systematisieren I -- Typologien}: Die Flexibilität, Adaptivität und Komplexität von Mixed Methods impliziert eine ungeheure Fülle an Möglichkeiten, wie ein konkretes Projekt aussehen kann. Typologien haben sich als eine von drei zentralen Wegen erwiesen, Ordnung in diese Vielfalt zu bringen. Entscheidend sind dabei die Kriterien, nach denen typologisiert wird. Die Identifikation eines passenden Typs von Mixed-Methods-Design kann helfen, einen realistischen Weg zur Umsetzung eines Projekts zu finden.
\item
  \textbf{Mixed Methods systematisieren II -- Notationen}: Notationen sind neben Typologien eine zweite Möglichkeit, Mixed Methods systematisch darzustellen. Im Vergleich zu Typologien legen Notationen stärkeres Gewicht auf die grundlegende Logik eines Projekts. Sie sind zudem leicht zu verstehen und damit auch leicht zu kommunizieren.
\item
  \textbf{Mixed Methods systematisieren III -- Visualisierung}: Mixed-Methods-Projekte sind in aller Regel mehrstufig -- sie sind also in mehrere Komponenten gegliedert, die häufig in einer (mehr oder weniger zwingendenden) logischen Reihenfolge abzuarbeiten sind. Visualisierungen helfen, den Überblick über diese verwobenen Prozesse zu behalten.
\item
  \textbf{Varianten der Methodenintegration}: Anhand konkreter Beispiele lassen sich die verschiedenen Varianten leicht erfassen, wie in einem Forschungsprozess Methoden sinnvoll kombiniert werden können. Eine der wichtigsten Entscheidungen betrifft den Zeitpunkt der Integration: während der Datenproduktion, in der Phase der Datenanalyse oder im Zuge der Präsentation und Rahmung von Ergebnissen.
\item
  \textbf{Mixed Methods ohne Mixing}: Für kleinere Forschungsprojekte -- etwa für eine Bachelor- oder Masterarbeit -- ist es kaum realistisch machbar, mehrere Forschungsmethoden konsequent einzusetzen. Das bedeutet aber nicht, dass man sich nicht von Mixed Methods inspirieren lassen kann. Es gibt verschiedene Wege, ``sekundäre'' Mixed-Methods-Projekte umzusetzen.
\item
  \textbf{Wissenschaftstheoretische Grundlagen -- ein Primer}: Im Vergleich zu traditionellen Feldern der Methodenentwicklung und Methodenlehre zeichnet sich das Feld der Mixed Methods auch durch eine erhöhte Sichtbarkeit von wissenschaftstheoretischen Fragen aus. Charakteristisch ist in diesem Zusammenhang, dass verschiedene epistemologische Programme als Bezugspunkte anerkannt werden. Ein grober Überblick h
\end{enumerate}

\hypertarget{part-grundlagen}{%
\part{Grundlagen}\label{part-grundlagen}}

\hypertarget{designsmethoden}{%
\chapter{Designs und Methoden}\label{designsmethoden}}

Schon ein oberflächlicher Blick in die Literatur zeigt, dass der Begriff des Designs in der Mixed-Methods-Forschung eine prominente Rolle spielt \citep[siehe etwa][]{kuckartz2014, creswell2018}. Diese Bedeutung hat er aus zumindest zwei Gründen. Erstens stellt sich bei der Kombination verschiedener Methoden fast zwangsläufig die Frage, auf welcher wissenschaftstheoretischen Grundlage Erhebungstechniken, Materialien und Analyeformen verbunden werden; es stellt sich, in anderen Worten, die Frage nach der \emph{Logik} eines Projekts. Zweitens sind Mixed-Methods-Projekte auch in der organisatorischen Umsetzung herausfordernd, weil es gilt, vielfältige und verschiedene Arbeitspakete aufeinander abzustimmen und in einen bewältigbaren Ablauf zu bringen; Mixed Methods sind auch eine Frage der \emph{Logistik}.

Weil diese Fragen für Mixed Methods so eine zentrale Bedeutung haben, wird im Folgenden der Begriff des Forschungsdesigns ein wenig detaillierter besprochen. Dieser verweist auf eine Reihe von entscheidenden Kompetenzen, die über das Feld von Mixed Methods hinaus für die empirische Sozialforschung nützlich sind.

\hypertarget{zwei-kompetenzen-werkzeuge-und-wege-empirischer-forschung}{%
\section{Zwei Kompetenzen: Werkzeuge und Wege empirischer Forschung}\label{zwei-kompetenzen-werkzeuge-und-wege-empirischer-forschung}}

In seiner heute üblichen Bedeutung bezeichnet der Begriff ``Methode'' ein zielorientiertes und systematisches Vorgehen. Aus diesem Begriffsverständnis lassen sich zwei Metaphern ableiten, die Herausforderungen und Aufgaben methodisch stringenter Forschung auf den Punkt bringen. Diese beiden Metaphern helfen, Methodenfragen im engen Wortsinn von Designfragen zu unterscheiden:

\textbf{Erstens das Bild von einer Methode als Instrument oder Werkzeug.} Häufig implizit, häufig aber auch ausdrücklich ist dieses Bild in der Methodenliteratur weit verbreitet. Die Rede ist dann etwa vom Werkzeugkasten der empirischen Sozial- und Bildungsforschung. Der Wert der Metapher von Methoden als Werkzeugen ist, dass sie den Blick auf wichtige Aspekte und Probleme empirischer Forschung lenkt: Ist das Werkzeug \textbf{angemessen} -- passt es also zum \textbf{Material} oder, umgekehrt, welches Werkzeug passt zu meinem \textbf{Gegenstand}? Wie funktioniert das Werkzeug? Wie handhabe ich es korrekt? Die Metapher funktioniert auch als Warnung vor ``Maslows Hammer'', auch als ``Law of the Instrument'' bekannt: ``If all you have is a hammer, everything looks like a nail!'' Auch soll ein Werkzeug natürlich nie zum Selbstzweck werden. Die Entwicklung und Verfeinerung von Werkzeugen ist notwendig und lohnend, aber in letzter Instanz muss eine Forschungsmethode verstanden als Werkzeug dem übergeordneten Ziel dienen, unser Verständnis sozialer Welten zu fördern.

Neben der Werkzeugmetapher können wir die wörtliche Übersetzung von \textbf{Methode als ``Vorgehen'' aber zweitens auch stärker in die Richtung des Bilds von einem ``Weg zu einem Ziel'' fassen.} Dieses Bild nimmt die ursprüngliche Bedeutung des altgriechischen ``methodos'' wörtlich, das sich mit ``Weg zu etwas hin'' übersetzen lässt. Als Metapher ernst genommen, lenkt dieses Bild die Aufmerksamkeit auf andere Aspekte eines gelingenden Forschungsprozesses als die Vorstellung von Methoden als Werkzeugen: Zunächst muss ein \textbf{Ziel} definiert werden, und das möglichst unmissverständlich, gut kommunizierbar und klar (das ist die Rolle von \protect\hyperlink{forschungsfragen}{Forschungsfragen}). Dann geht es darum, mögliche Wege zu diesem Ziel zu skizzieren, wahrscheinliche \textbf{Hindernisse und Hürden} im Forschungsprozess zu identifizieren sowie entscheidende Kreuzungspunkte und schwierige Wegstellen zu antizipieren. Je mehr damit zu rechnen ist, dass im Zuge der Reise unvorhergesehene Situationen bewältigt und Entscheidungen getroffen werden müssen, desto mehr werden wir abstrakte und möglichst griffige \textbf{Regeln und Kriterien} benötigen, die vor Irrtümern, Umwegen und Sackgassen bewahren. Im Vergleich zur Werkzeugmetapher verweist der buchstäblich und bildlich genommene ``Weg zu einem Ziel'' auf mehr und vielfältigere Facetten, ist aber zumindest für die Planungsphase auch grobschlächtiger. Wie bei einer Landkarte wird der Forschungsprozess \textbf{aus der Vogelperspektive} in den Blick genommen.

Die beiden Metaphern bringen zwei \textbf{Grundkompetenzen} für empirische Forschung zum Ausdruck. Ausgehend von der Unterscheidung dieser beiden Kompetenzen wird in der aktuellen Literatur zwischen \textbf{Methodenfragen} im engen Sinn und \textbf{Designfragen} unterschieden. Die Rede von zwei Grundkompetenzen ist einerseits als ein Versprechen zu lesen: Wer diese Fertigkeiten ernst nimmt und schult, wird empirische Forschungsprojekte mit größerer Überzeugung, mehr Spaß und mehr Erfolg durchführen. Sie impliziert andererseits aber auch zwei Formen der \textbf{Bewertung} bzw. der potenziellen \textbf{Kritik} an empirischen Forschungsprojekten: Das wissenschaftliche Fachpublikum wird ein Projekt stets auf beiden Ebenen bewerten -- im Hinblick auf seine Grundlogik, seine übergreifende Kohärenz und Schlüssigkeit (Designperspektive) ebenso wie hinsichtlich der sauberen Umsetzung im Detail, hinsichtlich seiner Rigidität und der sauberen Anwendung von Werkzeugen (Methodenperspektive).

Die Unterscheidung von Methoden- und Designfragen ist instruktiv und wichtig. Sie darf aber nicht als Dichotomie missverstanden werden. Methoden- und Designfragen sind nicht scharf zu trennen. Eher handelt es sich um zwei unterschiedliche Perspektiven auf dieselben Prozesse. Jede Methodenfrage wird entsprechend Designaspekte aufweisen, und jede Designfrage kann nur sinnvoll in Kombination mit im engen Sinn methodischen Aspekten diskutiert werden.

\hypertarget{drei-grunddesigns}{%
\section{Drei Grunddesigns}\label{drei-grunddesigns}}

Empirische Forschung zielt auf Verallgemeinerbarkeit. Im Gegensatz zu anderen Praxisfeldern zielen die Erkenntnisinteressen der (Sozial-)Wissenschaften fast immer über die konkret beforschten Fällen und Situationen hinaus. Die ``logische'' Funktion eines Forschungsdesigns liegt darin, sicherzustellen, dass ein Projekt in diesem Sinne \textbf{generalisierbare Aussagen} zulässt.

Für die Sozialwissenschaften lassen sich grob drei Formen der Generalisierung untersdcheiden, die den \textbf{drei Grunddesigns sozialwissenschaftlicher Forschung} entsprechen \citep{devaus2001}: Experiment, Erhebung und Fallstudie. Diese drei Designs zeichnen sich durch ihre je eigene Strategie aus, zu verallgemeinerbaren Aussagen zu kommen \citep{polit2010}. Jedes der Designs erfordert, dass bestimmte Vorannahmen getroffen werden und bestimmte Voraussetzungen erfüllt sind. Jedes der Designs geht mit je eigenen Qualitätskriterien und Geltungsbedingungen einher. Und jedes trägt auf seine Art zur Herstellung eines Anwendungsbezugs von Forschung bei:

\textbf{Experimente} verfolgen als Generalisierungsziel die Quantifizierung von Effekten \citep{kirk2013}. In der Isolation von Effekten besteht die methodologische Kernherausforderung. Sie erfordert (idealerweise) unter Laborbedingungen durchgeführte, vollständig randomisierte Kontrollgruppendesigns. Diese Anforderungen sind in den Sozialwissenschaften selten in Reinform zu erfüllen, schon alleine, weil eine komplett zufällige Gruppeneinteilung und eine totale Kontrolle über die Implementationsbedingungen nur schwer zu realisieren sind. In der Sozialforschung wird man daher eher auf quasi-experimentelle Designs stoßen. Quasi-Experimente sind der Art des Erkenntnisanspruchs nach Experimente, ihrer Form nach aber nur zum Teil.

Experimentelle Designs lassen für sich genommen keine Schlüsse von den untersuchten Fällen auf eine breitere Population, keine Aussagen über die konkreten Mechanismen und Faktoren, die zur Wirksamkeit einer Intervention führen, und keine Einschätzung der realen Effekte einer Maßnahme unter ``Realbedingungen'' zu.

\textbf{Erhebungen} verfolgen als Generalisierungsziel den Schluss von einer Stichprobe auf eine Grundgesamtheit \citep{dillman2014}. Es geht also um Repräsentativität oder auch um externe Validität (im Kontrast zur internen Validität von Experimenten). Um gültig zu sein, muss dieser Schluss auf Basis von unter Realbedingungen und randomisiert gezogenen Fällen aus einer klar definierte Grundgesamtheit erfolgen.

Erhebungen erlauben für sich genommen keine Aussagen über Kausalitäten. Sie können Informationen zu relevanten hinderlichen und förderlichen Faktoren liefern, aber nicht die zugrundeliegende Wirkrichtung und auch nicht die involvierten Mechanismen benennen. Im Vergleich zu Experimenten decken Erhebungsdesigns üblicherweise eine größere Bandbreite an Faktoren und Aspekten ab. Mit Experimenten teilen sie aber die Einschränkung, dass die wesentlichen Begriffe und Konzepte einer Studie vorab definiert und operationalisiert sein müssen. Entdeckende und theoriebildende Forschung ist damit nur bedingt möglich.

\textbf{Fallstudien} verfolgen als Generalisierungsziel die umfassende Bestimmung der Faktoren und Mechanismen, die in ihrem Wechselspiel ein Phänomen konstituieren, über den Weg der systematischen Analyse eines Problems in seiner komplexen Ganzheit \citep{yin2014, ylikoski2018}. Um tragfähige Aussagen machen zu können, muss erstens die Auswahl der untersuchten Fälle theoretisch und empirisch begründet sein und muss zweitens durch kombinierten Methodeneinsatz dafür gesorgt werden, dass die (dem Anspruch nach: alle) wesentlichen Aspekte identifiziert und in der Analyse berücksichtigt werden.

Fallstudien erlauben für sich genommen keine Aussagen über Stärke von Effekten und über die Häufigkeit von bestimmten Konstellationen.

\textbf{Mixed-Methods-Forschung bedeutet bei genauerer Betrachtung häufig (aber nicht immer) auch Mixed-Design-Forschung!} In einem typischen Mixed-Methods-Projekt kann ein Design in ein anderes eingebettet werden (beispielsweise eine Fragebogenerhebung als Teil einer Fallstudie implementiert werden) oder es können auch mehrere Designs parallel eingesetzt werden (zum Beispiel, wenn ein Experiment durchgeführt wird, in dem die Wirksamkeit einer Unterrichtsmethode ``unter Laborbedingungen'' getestet wird, und zusätzlich in einer Fallstudie die Implementation derselben Maßnahme in ``natürlichen'' Unterrichtssettings umfassend untersucht wird).

\hypertarget{was-das-alles-fuxfcr-mixed-methods-bedeutet}{%
\section{Was das alles für Mixed Methods bedeutet \ldots{}}\label{was-das-alles-fuxfcr-mixed-methods-bedeutet}}

Die Bedeutung von Designfragen macht sich im Feld der Mixed-Methods-Forschung vielfältig bemerkbar. So ist es letztlich die Grundaufgabe von \protect\hyperlink{typologien}{Typologien}, häufige Arbeitsabläufe und -pakete systematisch darzustellen und zu reflektieren. Auch die typischen \protect\hyperlink{notationen}{Notationssysteme} erfüllen letztlich eine Designfunktion: die Grundlogik eines Projekts abzubilden.

Um ein schlüssiges und fruchtbares Mixed-Methods-Design zu erarbeiten, wird man speziell darauf achten müssen, \ldots{}

\ldots{} dass die Forschungsfrage die Projektanliegen passend widerspiegelt,

\ldots{} dass der Gegenstand / die Gegenstände des Projekts geklärt sind,

\ldots{} dass reflektiert wird, welche Generalisierungsziele verfolgt werden und welche Rolle die Logiken der drei Grunddesigns spielen,

\ldots{} dass der Stichprobenlogik bzw. der Kombination verschiedener Stichprobenverfahren Aufmerksamkeit geschenkt wird,

\ldots{} dass klar ist, welche Methoden zum Einsatz kommen,

\ldots{} dass festgelegt ist, wie die verschiedenen Methoden zueinander stehen (Reihenfolge, Priorität) und

\ldots{} dass jede Designkomponente (jede Methode, jedes Arbeitspaket) ein klares Ziel verfolgt.

\hypertarget{qualquan}{%
\chapter{Qualitativ versus quantitativ?}\label{qualquan}}

Unterscheidung definiert das Feld

Überraschend unklar und umstritten

\hypertarget{eine-irrefuxfchrende-unterscheidung}{%
\section{Eine irreführende Unterscheidung}\label{eine-irrefuxfchrende-unterscheidung}}

Häufige Erzählung:
Eine Spalte: nomologisch, quantitativ, positivistisch, erklärend
Andere Spalte: hermeneutisch, konstruktivistisch, verstehend, interpretativ

Wird den Unterschieden innerhalb der Felder nicht gerecht

Wird den Überlappungsbereichen nicht gerecht

Keine scharfe Trennlinie, auf die sich alle einigen könnten

Zusammenhängende Aspekte: Wissenschaftstheoretische Bezüge

\hypertarget{eine-sinnvolle-unterscheidung}{%
\section{Eine sinnvolle Unterscheidung}\label{eine-sinnvolle-unterscheidung}}

Bestehende und gewachsene Traditionen und Felder; Vorstellungen sinnvoller und guter Forschung, Glaubwürdigkeit, Relevanz, Ziele

Erfahrungsschatz und Erwartungshorizonte

Sich mit diesen verschiedenen Logiken, zu forschen, vertraut machen

Hilft, Ziele konkreter zu definieren

Hilft, über Gegenstände nachzudenken

Hilft, über Forschungsprozess nachzudenken

Epistemische Ziele für qualitative Methoden

Verstehen

Typologisieren

Rekonstruieren

Phänomene in ihrer Komplexität erschließen

Epistemische Ziele für quantitative Methoden

Häufigkeit und Verteilung

Monitoring und Veränderungen über die Zeit

Isolierte Kausalzusammenhänge

Gegenstände

In allen relevanten Hinsichten bekannt / In Grundzügen erst zu erschließen

Als in dimensionaler Struktur stabile und bekannte Konstrukte gedacht / Als komplexe, kontextualisierte und wandelbare Phänomene gedacht

Probleme, Prozesse und Prinzipien

Herausforderungen, die mit diesen Zielen verbunden sind und Prinzipien, nach denen Forschungsprozesse gestaltet und bewertet werden

Forschungsstrategien für qualitative Methoden

Flexibel

Zyklisch

Begründet

Forschungsstrategische Prinzipien für quantitative Methoden

Standardisiert

Linear

Randomisiert

\hypertarget{weil-und-um-zu}{%
\section{``Weil'' und ``Um zu''}\label{weil-und-um-zu}}

Weil-Motive: die Methodenwahl ergibt sich aus dem Gegenstand

Qualitativ, weil keine Standardisierung möglich ist

Quantitativ, weil Vergleichbarkeit über Kontexte und Zeitpunkte hinweg angestrebt wird

Um-zu-Motive: die Methodenwahl informiert die Art und Weise, wie der Gegenstand konstruiert wird

Qualitativ, um einen quantitativen Befund zu validieren

Qualitativ/quantitativ, um eine quantitative/qualitative Studie zu ergänzen

Quantitativ, um die Verteilung von qualitativ identifizierten Mustern zu festzustellen

\hypertarget{gegenstaendeziele}{%
\chapter{Gegenstände und Ziele bestimmen}\label{gegenstaendeziele}}

Um aus einer Forschungsfrage abzulesen, welche Methoden und welches Design für ein Projekt geeignet sind, müssen zwei Punkte deutlich sein: Gegenstand und Erkenntnisziele. Wir sich über diese beiden Aspekte eines Forschungsvorhabens im Klaren ist, kann bewusste und sinnvolle Entscheidungen treffen.

Beide Punkte sind alles andere als trivial. Der konkrete Gegenstand eines Forschungsprojekts kann oft nur umschrieben werden. Und Ziele sind in aller Regel mehrstufig und mehrschichtig. In beiden Fällen geht es darum, über das eigene Projekt nachzudenken. Dafür gibt es kein endgültiges, vorgefertigtes Schema.

\hypertarget{thema-vs.-gegenstand-vs.-material}{%
\section{Thema vs.~Gegenstand vs.~Material}\label{thema-vs.-gegenstand-vs.-material}}

Jede empirische Forschung hat ihren Gegenstand. Der Gegenstand eines Projekts ist nicht mit seinem Thema und auch nicht mit dem empirischen Material zu verwechseln. Er hat aber selbstverständlich mit beidem viel damit zu tun.

Beispielsweise kann sich ein Projekt dem \emph{Thema} ``Rassismus in Universitätskontexten'' widmen. Das Thema definiert hier den inhaltlichen Problembereich. Der \textbf{Gegenstand} bezeichnet demgegenüber die Art von sozialem Phänomen, die untersucht werden soll. Zum Beispiel könnte unter der breiten Thema nach \emph{Erfahrungen mit rassistischer Diskriminierung} gefragt werden; das wäre ein anderer Gegenstand, als wenn für dasselbe Thema \emph{diskriminierende Verhaltensweisen} untersucht werden sollen; ein dritter möglicher Gegenstand zum selben Thema wäre definiert, wenn unter Rassismus \emph{überindividuelle Wissensordnungen} verstanden werden.

Je nach empirischen Gegenstand sind verschiedene Arten \textbf{empirischen Materials} unterschiedlich angemessen. Für den Gegenstand rassistische Erfahrungen könnten unter Umständen mit Fragebögen und/oder qualitativen Interviews gearbeitet werden; um diskriminierende Verhaltensweisen zu untersuchen, müsste idealerweise beobachtet oder mit Feldexperimenten gearbeitet werden; um überindividuelle Wissensordnungen (``Diskurse'') zu erschließen, könnten etwa Dokumente analysiert werden.

Realistisch nutzbare Materialien werden dem angestrebten Gegenstand nie perfekt entsprechen. Der Gegenstand mag beispielsweise rein logisch Beobachtungen erfordern (Praktiken, Verhalten), aus organisatorischen Gründen muss aber mit Interviews gearbeitet werden. Solche Brüche zwischen Gegenstand und Material sind häufig nicht zu vermeiden. Wichtig ist, sie offen darzustellen und ihre Folgen für die Aussagekraft eines Projekts zu reflektieren. Mixed Methods können in solchen Settings hilfreich sein, um einen schwer empirisch zu fassenden Gegenstand zumindest aus unterschiedlichen Blickwinkeln betrachten zu können.

Gerade in Mixed-Methods-Studien kann es vorkommen, dass verschiedene Gegenstände erforscht werden sollen. Zum Rassismus an Hochschulen könnten so zum Beispiel sowohl Erfahrungen mit Diskriminierung als auch diskriminierende Verhaltensweisen untersucht werden, um sie dann zueinander in Beziehung setzen zu können. Wichtig ist nur, beide Gegenstände und ihre Implikationen für die Gestaltung des Projekts bewusst vor Augen zu haben.

\hypertarget{erkenntnisziele}{%
\section{Erkenntnisziele}\label{erkenntnisziele}}

Während die Bestimmung des Gegenstands primär der Methodenwahl dient, hat die Definition von Erkenntniszielen in erster Linie Folgen für das Forschungsdesign. Für Mixed-Methods-Projekte sollten Ziele auf zumindest drei Stufen festgelegt werden:

\textbf{1. Welche Art von Generalisierungsziel wird angestrebt?} Auf dieser allgemeinsten Ebene geht es darum, den Typ von wissenschaftlicher Zielsetzung zu klären. Aus dieser lässt sich ablesen, welche(s) sozialwissenschaftliche(n) Grunddesign(s) angemessen sind. Soll, um beim Beispiel Rassismus an Hochschulen zu bleiben, die Häufigkeit von Rassismuserfahrungen erfasst werden, liegt ein \textbf{deskriptives Erkenntnisinteresse} vor, das auf \textbf{Repräsentativität} abzielt. Forschungslogisch liegt ein Erhebungsdesign nahe. Soll erkundet werden, wie Erfahrungen mit Rassismus Selbstbilder und Lebenswege prägen, dann liegt ein ein \textbf{exploratives Erkenntnisinteresse} vor, das darauf abzielt, ein soziales Phänomen \textbf{in seiner ganzen Komplexität} auszuloten. Das Grunddesign der Wahl ist demnach eine Fallstudie. Soll erforscht werden, wie sich Formen der medialen Berichterstattung auf Einstellungen auswirken, dann liegt ein \textbf{explanatives}/\textbf{analytisches} Erkenntnisinteresse vor, das auf die \textbf{saubere Isolation eines Ursache-Wirkungs-Zusammenhangs} abzielt. Als Grunddesign ergibt sich das (Quasi-)Experiment.

\textbf{2. Welche Art von Erkenntnissen/Ergebnissen zum konkreten Gegenstand können erwartet werden?} Die Bestimmung von epistemischen Zielen ist wichtig, um die Kohärenz eines Projekts sicherzustellen. Sie bleibt aber \textbf{abstrakt} und hilft wenig für die Beschäftigung mit dem \textbf{konkreten Phänomen} bzw. der \textbf{inhaltlichen Problemstellung}. Es empfiehlt sich daher, auf einer zweiten Ebene die inhaltlichen, gegenstandsspezifischen Ziele eines Projekts zu formulieren. In experimentellen Studien läuft das letztlich auf die Formulierung von \textbf{Hypothesen} hinaus. In Fallstudien und Erhebungen wird es eher darum gehen, zu umschreiben, \textbf{welche \emph{Art} von Aussagen und Ergebnissen} erwartet wird. Wird auf Ebene 1 die logische Kohärenz eines Projekts verhandelt, geht es auf dieser zweiten Ebene um die Frage der inhaltlichen \textbf{Relevanz}.

\textbf{3. Welches Ziel wird mit jeder der einzelnen Projektkompenten / Methoden verfolgt?} Speziell in Mixed-Methods-Studien wird es notwendig sein, auf einer dritten Ebene über Ziele nachzudenken. Auf dieser dritten Ebene werden die einzelnen Arbeitspakete / Projektkomponenten fokussiert: Für jede Methode und jede Komponente muss so explizit wie möglich definiert werden, welches (Zwischen-)Ziel mit ihr verfolgt wird. So kann beispielsweise eine qualitative Interviewstudie geführt werden, \emph{um eine Grundlage für die Formulierung von Fragebogenitems zu erarbeiten}. Oder es wird eine Fragebogenerhebung durchgeführt, \emph{um aufgrund erster quantitativer Verteilungen eine Basis für die begründete Auswahl von Interviewpartner/innen zu haben}. Diese beiden Beispiele implizieren ein sequenzielles Mixed-Methods-Design -- das zweite Arbeitspaket kann erst angegangen werden, wenn das erste erledigt ist. Ein Beispiel für ein paralles Design wäre, wenn qualitative Interviews geführten werden, \emph{um die Validität eines Testinstruments zu validieren} oder \emph{um eine experimentelle Wirksamkeitsstudie mit einer ethnographischen Studie vertiefend zu begleiten}.

\hypertarget{forschungsfragen-und-ihre-funktionen-fuxfcr-die-mixed-methods-forschung}{%
\chapter{Forschungsfragen und ihre Funktionen für die Mixed-Methods-Forschung}\label{forschungsfragen-und-ihre-funktionen-fuxfcr-die-mixed-methods-forschung}}

In der Mixed-Methods-Literatur stößt man unausweichlich auf die Feststellung, dass die Forschungsfrage ausschlaggebend für die \protect\hyperlink{designsmethoden}{Methoden- und Designwahl} sein soll \citep[Kap. 5]{tashakkori2007, tashakkori2010, creswell2018}: ``{[}R{]}esearch questions in mixed methods studies are vitally important because they, in large part, dictate the type of research design used, the sample size and sampling scheme employed, and the type of instruments administered as well as the data analysis techniques'' \citep[475]{onwuegbuzieleech2006}.

Dieses prinzipielle Bekenntnis zum Stellenwert von Forschungsfragen findet sich auch in anderen methodologischen Zusammenhängen, wird für Mixed Methods aber ganz grundlegend programmatisch.\footnote{Siehe zum Beispiel: \url{https://f.hypotheses.org/wp-content/blogs.dir/6542/files/2019/05/MMR5.jpg} und zahlreiche andere Online-Ressourcen zum Thema.}

Doch wieso sind Forschungsfragen eigentlich so wichtig? Wie genau liest man aus einer Frage ab, welche Methoden zu ihrer Beantwortung notwendig sind? Und umgekehrt: Worauf ist bei der Formulierung einer Forschungsfrage zu achten, wenn sie eine kohärente Methodenwahl zulassen soll?

\hypertarget{was-macht-eine-frage-zur-forschungsfrage}{%
\section{Was macht eine Frage zur Forschungsfrage?}\label{was-macht-eine-frage-zur-forschungsfrage}}

Was genau ist aber nun eigentlich eine Forschungsfrage? Als erste Annäherung können wir definieren, dass eine Forschungsfrage Anspruch, Grundlage und Breite eines Forschungsvorhabens widerspiegelt. Mit Anspruch ist hier die Art des Erkenntnisanspruchs gemeint: Welche Art von Beitrag zu aktuellen wissenschaftlichen Debatten verspricht das Projekt? Mit Grundlage sind in erster Linie theoretische Bezugspunkte, in zweiter Linie die empirische Basis gemeint. Und mit Breite wird letztlich auf das Problem der Verallgemeinerbarkeit abgezielt: Sind die Befunde auf einen engen Teilausschnitt der sozialen Welt beschränkt bzw. zu welchem Grad sind sie auf andere als die direkt beforschten Kontexte übertragbar?

\hypertarget{die-vier-funktionen-von-forschungsfragen}{%
\section{Die vier Funktionen von Forschungsfragen}\label{die-vier-funktionen-von-forschungsfragen}}

Dass eine ``gute Forschungsfrage'' für das Gelingen eines Forschungsprojekts zentral ist, wird kaum jemand bezweifeln. Sehr viel weniger klar ist, worin ihre Bedeutung eigentlich genau besteht. Das mag daran liegen, dass Forschungsfragen nicht eine logisch notwendige und klar umrissene Funktion erfüllen. Die herausragende Relevanz von Forschungsfragen ergibt sich vielmehr aus dem Umstand, dass sie mehrere Funktionen gleichzeitig erfüllen. Forschungsfragen sind die Antwort auf verschiedene Probleme, die sich im Kontext empirischer Forschung stellen.

Grob lassen sich vier solcher konkreter Probleme benennen. Alle vier sind für Mixed Methods von spezieller Relevanz:

\textbf{1. Forschungsfragen schlagen eine Brücke zwischen Theorie und Empirie}: Eine gute Forschungsfrage bezieht sich auf wissenschaftliche Literatur, um ein Rätsel zu formulieren, das nur durch aktive Forschungsbemühungen zu ``lösen'' ist. Sie steht daher immer mit einem Bein in der Theorie (verwendet zum Beispiel theoretische Begriffe und Konzepte), mit dem anderen in der Empirie (lässt erkennen, was zu tun ist, um beantwortet zu werden). Diese Brückenfunktion macht einen großen Teil der ominösen Macht von Forschungsfragen aus, die Methodenwahl zu bestimmen.

\textbf{2. Forschungsfragen definieren Erkenntnisziele}: Aus einer gut formulierten Forschungsfrage lassen sich die Erkenntnisziele eines Projekts ablesen. Für Mixed-Methods-Projekte sind klar definierte Ziele entscheidend, weil nur so Prioritäten gesetzt und einzelne Komponenten sinnvoll (=zielorientiert) aufeinander abgestimmt werden können.

\textbf{3. Forschungsfragen helfen, Entscheidungen zu treffen}: Die Definition Erkenntniszielen ist auch deshalb wichtig, weil empirische Forschung ständig Entscheidungen verlangt. Ohne einen Kompass in Form klarer Forschungsfragen läuft man Gefahr, diese Entscheidungen inkohärent zu treffen -- also von Situation zu Situation nach unterschiedlichen Kriterien zu entscheiden. Diese vielen kleinen Gestaltungsschritte können sich schnell zu einem undurchsichtigen Gesamtgebäude fügen, in dem man sich nicht orientieren kann. Für Mixed Methods wiegt das noch einmal schwerer, weil hier mehrere empirische Projektteile aufeinander abgestimmt werden müssen.

\textbf{4. Forschungsfragen erfüllen eine Kommunikationsfunktion}: Eine häufig unterschätzte Funktion von Forschungsfragen liegt darin, die Anliegen eines Projekts einfach und verständlich kommunizieren zu können. Eine klar formulierte Frage lässt (gerade weil sie kurz und auf das Wesentliche reduziert ist) viel weniger Spielraum für Missverständnisse als andere Formen der Projektdarstellung. Das ist gerade für Mixed-Methods-Projekte entscheidend, weil sich in diesen Missverständnisse zusätzlich aus den verschiedenen methodologischen Hintergründen von Gesprächspartner*innen ergeben können.

Diese Grundfunktionen von Forschungsfragen hängen eng zusammen. In ihrer Summe laufen sie darauf hinaus, dass Forschungsfragen ein Format zum konzentrierten Nachdenken über ein empirisches Forschungsprojekt sind: In ihnen laufen Ziele, Forschungsstrategien, theoretische Bezugspunkte, Ideen zur empirischen Umsetzung und wissenschaftliche Bezugspunkte mit den alltäglichen Sorgen der konkreten Realisierung eines Projekts zusammen.

\hypertarget{wie-bestimmt-eine-forschungsfrage-die-methodenwahl}{%
\section{Wie bestimmt eine Forschungsfrage die Methodenwahl?}\label{wie-bestimmt-eine-forschungsfrage-die-methodenwahl}}

Die Erfahrung zeigt, dass es erstaunlich schwierig ist, eine zufriedenstellende Forschungsfrage zu formulieren. Woran merkt man, dass die Forschungsfrage ihren Zweck erfüllt? Wann habe ich eine Forschungsfrage, die als Richtschnur für ein empirisches Projekt funktionieren kann?

Die knappe Antwort lautet: Dann, wenn sich \protect\hyperlink{gegenstaendeziele}{der Gegenstand und die Ziele} eines Forschungsprojekts klar erkennen lassen.

Wann das der Fall ist, ist nicht immer leicht zu sagen. Eine gute Forschungsfrage soll leicht verständlich (und damit kommunizierbar) und nicht zu komplex sein (um auch wirklich Entscheidungen anleiten zu können). Daher ist es in der Regel nicht zielführend, Gegenstand und Ziele möglichst umfassend und präzise in der Frage auszudefinieren. Damit würde einfach schwerer zu handhaben und ihren praktischen Nutzen verlieren.

Hier sind drei Beispiele Forschungsfragen, die zwar schnell formuliert sind, aber sowohl Gegenstand als auch Ziele widerspiegeln und so Anhaltspunkte zur Methodenwahl liefern:

\emph{Wie verändert sich das Lernverhalten von Schüler*innen durch die Nutzung digitaler Lernmedien?} Primärer \emph{Gegenstand} dieser Frage ist das Lernverhalten. Dieser Begriff sollte für eine ``ernst gemeinte'' Forschungsfrage theoretisch unterfüttert sein und sich in einen breiteren Theorierahmen fügen. Aus dieser ersten Gegenstandsbestimmung ergibt sich, dass zur Beantwortung der Frage empirische Materialien nötig sind, die Verhalten widerspiegeln. Es geht also nicht darum, was Schüler*innen glauben oder sagen, sondern um das, was sie tatsächlich tun. Es geht auch nicht um Verhalten im Labor, sondern unter realen Bedingungen. Ideal wäre also eine Kombination aus Beobachtungen und Prozessmaterial. Die \emph{Zielsetzung} ist explorativ: Die Wie-Frage setzt voraus, dass sich etwas verändert, gibt aber keinerlei Hinweise auf die Hinsichten und die Richtung der Veränderung. Entsprechend scheint der Schwerpunkt auf qualitativen Methoden zu liegen, die von Offenheit und Flexibilität geprägt sind. Quantitative Methoden scheinen von untergeordneter Bedeutung, können aber wichtig werden, um beispielsweise wiederkehrende Muster zu identifizieren (z.B. dass Schüler*innen engagierter sind, wenn digitale Aufgaben eine bestimmte Komplexitätsstufe haben).

\emph{Welche Folgen hat die Umstellung auf Homeschooling für Bildungsungleichheiten in der Schweiz?} Diese Frage definiert als übergeordneten Gegenstand Bildungsungleichheiten. Auch das ist als ein theoretischer Begriff zu verstehen, der je nach Bezugsliteratur unterschiedlich verstanden wird. Die Zielsetzung hat eine deskriptive Schlagseite, wenn auch ein gewisser Kausalitätsanspruch mitschwingt: Es geht um die Beschreibung von Bildungsungleichheiten über die Zeit, wobei unterstellt wird, dass die Umstellung auf Homeschooling als Zeitreihenbruch behandelt werden kann. Die Frage definiert damit ein Erhebungsdesign. Sowohl Gegenstand als auch Zielsetzung legen vorwiegend quantitative Methoden nahe (es geht um die Veränderung von Ungleichheiten in einer großen Population). Welche Art von Daten geeignet ist, hängt von der genauen Definition von ``Bildungsungleichheiten'' ab - möglich wären beispielsweise Fragebögen zur Veränderung von Selbstbildern, Lebensentwürfen und Lernsituationen der Schüler*innen oder auch Kompetenztests. Qualitative Methoden könnten entweder in einer explorativen Vorstudie zum Einsatz kommen, beispielsweise als Grundlage für die Fragebogenentwicklung. Oder auch in einer vertiefenden Studie zum besseren Verständnis von Mustern aus der statistischen Datenanalyse.

\emph{Wie wirken sich interkulturelle Kontakte auf migrationspolitische Einstellungen aus?} Diese Frage definiert zwei Gegenstände: interkulturelle Kontakte und migrationspolitische Einstellungen. Der Schwerpunkt wird in der Formulierung eher auf letztere gelegt. Einstellungen sind Wissensformen. Methodisch liegt hier also kommunikative Forschungsformen nahe -- Interviews oder Fragebögen. Die Zielsetzung ist kausal-explanativ, wie das Verv ``auswirken'' signalisiert. Damit liegt ein (quasi-)experimentelles Design nahe. Die ``interkulturellen Kontakte'' wären in diesem Fall das ``Treatment'' (und natürlich noch genauer zu spezifizieren).

\hypertarget{darf-ich-meine-forschungsfrage-veruxe4ndern}{%
\section{Darf ich meine Forschungsfrage verändern?}\label{darf-ich-meine-forschungsfrage-veruxe4ndern}}

Die eigentlich gut gemeinte Feststellung, dass sich Methoden aus der Forschungsfrage ergeben sollen, kann leicht zu Verunsicherungen führen. Eine der häufigsten betrifft die Möglichkeit, im Fortgang eines Forschungsprojekts an der Fragestellung zu schrauben. Ist das vertretbar, wenn sich doch alle davor getroffenen Entscheidungen eigentlich aus der Frage ergeben haben sollten? Kommen Methoden und Fragen dann nicht erst recht wieder in ein schiefes bis verkehrtes Verhältnis?

Die Antwort ist wieder knapp: Ja, die Forschungsfrage darf und soll verändert werden! Die Verwirrung ergibt sich aus einem zweischrittigen Missvertändnis:

\begin{enumerate}
\def\labelenumi{\arabic{enumi}.}
\item
  Das Verhältnis von Forschungsfrage und Methoden muss logisch stimmig sein, nicht chronologisch festgezurrt: Die gewählten Methoden sollen sich inhaltlich schlüssig aus der Frage ergeben, sie müssen deswegen aber nicht zeitlich später festgelegt werden. In vielen Fällen ist die Entscheidung für oder gegen eine Methode schon vorab getroffen. Der Auftrag lautet dann, eine zu diesen Methoden stimmige Frage zu formulieren. Das ist nur dann ein Problem, wenn Punkt 2 nicht beachtet wird.
\item
  Die Forschungsfrage soll sich mitsamt ihrer Gegenstands- und Zieldefinition aus dem Stand der Forschung (= Theorie) ergeben: Die Frage soll zur methodischen Vorgehensweise passen, sie soll aber auch ein Rätsel formulieren, das zur relevanten Bezugsliteratur und zu den inhaltlichen Themenstellungen und Konzepten des jeweiligen Forschungsfelds passt. Kritisch wird es, wenn Methodenpräferenzen zu Fragen führen, die eigentlich keine theoretische Relevanz haben: Methoden -\textgreater{} Forschungsfrage(n) ?! -\textgreater{} ?! Theorie. Idealerweise sollte das Passungsverhältnis sein: Methoden -\textgreater{} Forschungsfrage(n) \textless- Theorie.
\end{enumerate}

Werden die Funktionen von Forschungsfragen ernst genommen, führt eigentlich kein Weg daran vorbei, sie permanent mit dem Projekt mitwachsen zu lassen. In der Forschungsfrage soll sich, so gesehen, das Projekt in seiner aktuellen Form konzentriert widergespiegelt finden.

\hypertarget{kann-ich-mehr-als-eine-forschungsfrage-haben}{%
\section{Kann ich mehr als eine Forschungsfrage haben?}\label{kann-ich-mehr-als-eine-forschungsfrage-haben}}

Eine zweite Verunsicherung betrifft die Anzahl der Forschungsfragen. Gerade für Mixed-Methods-Projekte wirkt die Rede von DER Forschungsfrage unter Umständen einschüchternd. Muss es wirklich die eine einzige Frage sein -- oder können auch mehrere Fragen formuliert werden?

In diesem Fall ist keine knappe Antwort möglich. In vielen Projekten wird es sinnvoll sein, verschiedene Frageformen und Frageebenen zu unterscheiden. Die übergeordnete Forschungsfrage kann dann in mehrere Unterfragen übersetzt werden, die jeweils eine Komponente oder eine Phase des Gesamtprojekts definieren. Diese Fragen können dann noch weiter heruntergebrochen werden, um beispielsweise die Datenerhebung an ein paar konkreten Fragen auszurichten.

In der Mixed-Methods-Literatur findet sich manchmal der Ratschlag, für qualitative und quantitative Methoden jeweils eigene und getrennte Fragen zu formulieren. Ob das gut funktioniert, hängt vom Projekt und auch von den eigenen Vorlieben ab. Zusätzlich wird aber in der Regel die Formulierung einer (integrierend-übergeordneten) ``Mixed-Methods-Frage'' vorgeschlagen \citep[siehe etwa][480ff.]{onwuegbuzieleech2006}.

In vielen Fällen ist eine eindeutige Zuordnung von qualitativen und quantitativen Fragen aber nicht möglich und kann auch hinderlich sein. Wenn sich ein klares Frageschema entlang der Trennlinie qualitativ-quantitativ finden lässt, dann ist es gut. Wenn nicht, dann ist das auch kein Grund zur Panik. Wahrscheinlich würde eine solche Aufteilung dem konkreten Projekt nicht gerecht.

In jedem Fall empfehle ich, immer zu versuchen, eine einzige übergeordnete Forschungsfrage zu formulieren. Das hilft sehr dabei, sich die eigenen Ziele und Anliegen vor Augen zu halten.

\hypertarget{part-techniken-und-wege-der-konkreten-umsetzung}{%
\part{Techniken und Wege der konkreten Umsetzung}\label{part-techniken-und-wege-der-konkreten-umsetzung}}

\hypertarget{typologien}{%
\chapter{Mixed Methods systematisieren I: Typologien}\label{typologien}}

Es ist eine fast schon identitätsstiftende Eigenheit von Mixed Methods, Forschungsdesigns in \textbf{Typologien} einzuteilen \citep{creswell2003}. Der Begriff des Designs wird dabei deutlich konkreter gefasst als in der Unterscheidung der \protect\hyperlink{designsmethoden}{drei Grunddesigns sozialwissenschaftlicher Forschung}. Die Grundidee bleibt aber dieselbe: Es geht darum, methodologisch \emph{schlüssige}, empirisch \emph{produktive} und praktisch \emph{machbare} Wege aufzuzeigen, Erkenntnisziele zu einem bestimmten Forschungsgegenstand zu erreichen.

Die Gründe für die Neigung zur ``Typenbildung'' in der Mixed-Methods-Literatur sind vielfältig \citep[@][]{teddlie2006}. In ihr kommt natürlich zum einen die Komplexität und den Variantenreichtum von Mixed-Methods-Projekten zum Ausdruck, die (im Gegensatz zu einer Fragebogenerhebung oder einem Experiment) nicht nach einem fixen und immer gleichen Schema ablaufen können. Sie spiegelt aber auch eine Art von philosophischer Haltung wider, nach der nicht abstrakte normative Vorstellungen idealer Wissenschaft, sondern reale Forschungabläufe Ausgangspunkt methodologischer Reflexion sein sollten macht. Nicht zuletzt geht es auch um die Ausarbeitung einer gemeinsamen Sprache und geteilter Standards, darum, Richtlinien zu definieren, um Mixed-Methods-Projekte nachvollziehbar gestalten und bewerten zu können.

\hypertarget{mm-typologien-ein-beispiel}{%
\section{MM-Typologien: ein Beispiel}\label{mm-typologien-ein-beispiel}}

Wer sich in die Mixed-Methods-Literatur vertieft, kann den Eindruck gewinnen, dass es so viele verschiedene MM-Typologien wie MM-Autor*innen gibt. Manche Typologien unterscheiden relativ grobschlächtig eine Handvoll von typischen Forschungsszenarien, andere sind deutlich feingliedriger und identifizieren Dutzende Typen von MM-Designs.

Um die Grundidee zu verstehen, ist es sinnvoll, sich mit einer exemplarischen Typologie vertraut zu machen. Dazu bietet sich jene von Creswell \& Plano Clark \citep{creswell2018} an, die sich mittlerweile als eine Art Standard etabliert hat. Unterschieden werden sechs zentrale Arten von MM-Projekten:

\textbf{1. Convergent Parallel Design}: Qualitative und quantitative Methoden werden in diesem Design gleichzeitig (das heißt: in keiner logisch notwendigen Reihenfolge) eingesetzt, um den Forschungsgegenstand aus unterschiedlichen Perspektiven in den Blick nehmen zu können (der Begriff der \emph{Triangulation} kann daher auch als Synonym für diesen Typ der Methodenintegration gelten). Qualitative und quantitative Befunde werden in der Regel am Ende des Forschungsprozesses aufeinander bezogen und kontrastiert. Datenerhebung und Datenanalyse laufen also getrennt ab, die Integration von qualitativen und quantitativen Befunden findet ex-post statt.

\textbf{2. Explanatory Sequential Design}: In einem explanatorischen Design hat die quantitative Komponente Vorrang. Der Vorrang ist ein zeitlicher, nicht unbedingt einer der Gewichtung. Qualitative Methoden werden einer zweiten Forschungsphase genutzt, um Muster aus der ersten quantitativen Phase vertiefend zu untersuchen (in der Regel um sie zu \emph{erklären}, daher der Name des Designs). Quantitative Daten müssen also erhoben und analysiert sein, bevor der qualitative Part starten kann.

\textbf{3. Exploratory Sequential Design}: Vorrang hat in diesem Fall (wieder: von der Reihenfolge, nicht zwangsläufig vom Stellenwert her) die qualitative Komponente. Die Daten aus dieser müssen erhoben und analysiert sein, bevor quantitative Methoden genutzt werden, um beispielsweise die Häufigkeit von qualitativ eruierten Konstellationen festzustellen oder auch um Hypothesen zu testen, die sich aus der qualitativen Forschung ergeben. (Dieser Designtyp entspricht einem Szenario, das in der deutschsprachigen Methodenliteratur lange Zeit als einziges ``Mixed-Methods-Design'' ernsthaft Erwähnung gefunden hat: Auf eine explorative qualitative folgt eine hypothesenprüfende quantitative Studie.)

\textbf{4. Embedded Design}: In diesem Fall hat entweder der qualitative oder der quantitative Part klar Priorität, das jeweils andere Methodeninstrumentarium kommt nur ``eingebettet'' ins Spiel, quasi als Hilfsmittel zu klar umrissenen Zwecken. Ein Grenzfall eines solchen Designtyps wäre die Inklusion von sehr offenen Items in eine Fragebogenerhebung oder eine kleine numerische Auszählung in einer ansonsten durch und durch qualitativen Studie. Eine etwas ambitioniertere Variante wäre z.B. eine Fragebogenerhebung, die im Rahmen einer ethnographischen Studie stattfindet; oder eine qualitative Interviewstudie, die ein Laborexperiment begleitet.

\textbf{5. Transformative Design}: Dieser ``Designtyp'' durchbricht das bisherige Schema deutlich. Kennzeichnend für diesen Typ von MM-Projekt ist, dass es sozialkritsch/politisch motiviert ist und emanzipatorisch verändernd auf die soziale Welt einwirken will. Diese Zielsetzung ist klar normativ, muss aber sozialtheoretisch verankert sein und methodologisch reflektiert in Forschungsaktivitäten überführt werden.

\textbf{6. Multiphase Design}: Dieser letzte Designtyp soll all jene Projekte umfassen, die Aspekte verschiedener Designs aufgreifen und kombinieren. Multiphase-Projekte sind ihrer Anlage nach komplex und vielschichtig und bestehen entweder aus mehr als nur zwei ``Sequenzen''/Forschungsphasen oder kombinieren sequenzielle mit parallel ablaufenden Komponenten. In der Praxis folgen sehr viele Designs diesem Typ. Es handelt sich um eine Art ``Sonstige''-Kategorie, die für alle etwas anspruchsvolleren Projekte reserviert ist.

Das Schema von Creswell \& Plano Clark kann als ``typische Typologie'' gelten -- sie ähnelt anderen Mixed-Methods-Typologien und greift Anregungen aus diesen auf (wie viel die beiden von anderen gelernt haben, zeigt sich, wenn man frühe Versionen bzw. direkte Vorläufer ihrer Typologie mit der aktuellen Fassung vergleicht, siehe etwa Creswell et al.~2003).

Um die Finessen dieser und anderer Typologien zu verstehen, lohnt es sich, andere Vorschläge vergleichend anzusehen. So schlagen andere Autoren Typologien vor, die auch rein qualitative und rein quantitative Projekte umfassen. Ein Beispiel ist die ``Methods-Strand-Matrix'' \citep{teddlie2006}, die unter anderem für ihre \textbf{Conversion Designs} Beachtung verdient. Diese zeichnen sich dadurch aus, dass die Daten aus einer Komponente in die methodische Logik der anderen übersetzt werden; Teddlie und Tashakorrie sprechen vom \emph{Quantitizing} qualitativer Daten, wenn etwa Kategorien aus einer qualitativen Inhaltsanalyse in statistische Modellierungen einfliessen, und vom \emph{Qualitizing} quantitativer Daten, wenn umgekehrt statistische Ergebnisse zum Inhalt einer interpretativen Analyse werden.

\hypertarget{was-wirklich-zuxe4hlt-kriterien-der-typenbildung}{%
\section{Was wirklich zählt: Kriterien der Typenbildung}\label{was-wirklich-zuxe4hlt-kriterien-der-typenbildung}}

Design-Typologisierung kommen mit ihren eigenen Fallstricken: Soll eine solche wirklich systematisch sein, wird sie sehr groß und unübersichtlich; wird sie demgegnüber übersichtlich und handhabbar gehalten, kann sie schnell willkürlich wirken. So steht Typ 4 (``Transformative Design'') seltsam quer zu den anderen; kann ein MM-Projekt nicht sowohl ``transformativ'' als auch, beispielsweise, ``explanativ'' sein? Und wieso taucht die Frage der Normativität, die für transformative Designs absolut prägend ist, in den anderen Typen gar nicht auf?

Der Wert von Mixed-Methods-Typologien liegt daher bei näherer Betrachtung nicht so sehr in den identifizierten Typen selbst. Entscheidend sind vielmehr die Kriterien, nach denen diese gebildet werden. Die Typen selbst können hilfreich sein, um ein Projekt schnell und für ein Fachpublikum verständlich zu charaktisieren. Für die Gestaltung und Bewertung eines Projekts muss aber tiefer geschürft werden. Die Kriterien einer Typologisierung können dabei als Richtschnüre dienen.

Achtet man mehr auf die Kriterien denn auf die Typen, zeigen sich zwischen den vielen Mixed-Methods-Typologien erstaunliche Ähnlichkeiten. Zu den zentralen Kriterien, die immer wieder auftauchen, zählen:

\begin{itemize}
\item
  \textbf{Grund für Methodenkombination}: In einigen Typologien ist es das zentrale Kriterium, in anderen ist es implizit bedeutsam: Aus welchem Grund sollen überhaupt qualitative und quantitative Methoden kombiniert werden? Im Schema von Creswell \& Plano Clark ist dieser Aspekt unter der Oberfläche präsent, wenn beispielsweise in einem ``Sequential Explanatory Design'', qualitative Methoden genutzt werden, \emph{um zu} erklären was auf Basis von nur quantitativen Daten nur als deskriptive Korrelationen beobachtet werden kann. Bei anderen Autor*innen ist dieses Kriterium das übergeordnete und entsprechend tonangebend \citep[beispielsweise][]{greene1989, bryman2006}.
\item
  \textbf{Priorität}: Eines der zentralen Kriterien für viele Typologien ist der relative Stellenwert von qualitativen und quantitativen Methoden. Für jedes Projekt sollte also überlegt werden: Sprechen die \protect\hyperlink{gegenstaendeziele}{gegebenen Erkenntnisziele und der gegebene Gegenstand} für ein eher qualitatives oder ein eher quantitatives Projekt? Oder ergibt sich eher ein ausgewogenes Verhältnis?
\item
  \textbf{Integrationspunkte und Schnittstellen}: Dass qualitative und quantitative Methoden kombiniert werden, sagt noch nichts darüber aus, wie und an welcher Stelle im Forschungsprozess das passiert. Hier gilt es präzise zu sein: Soll der Gegenstand (wie in der klassischen Triangulation) aus verschiedenen -- das heißt eigenständigen und weitgehend getrennten -- Blickwinkeln ausgelotet werden? Dann findet die Integration in der Phase der abschließenden Interpretation oder Ergebnisdiskussion statt. Oder sollen die Ergebnisse eines Elemente für die Gestaltung eines anderen verwendet werden? Falls ja, wie und wo genau (soll die Information aus Phase 1 für die Gestaltung eines \emph{Erhebungsinstruments} oder eher für die \emph{Stichprobengestaltung} in Phase 2 genutzt werden)? Oder geht es darum, parallel an qualitativen und quantitativen Daten zu arbeiten, diese aber ständig aufeinander zu beziehen (etwa, wenn qualitative Interviews und Daten aus einer Fragebogenerhebung gleichzeitig ausgewertet werden und aktiv aufeinander bezogen werden)?
\item
  \textbf{Reihenfolge und Abhängigkeit}: Die Definition von Integrationspunkten und Schnittstellen hängt eng mit der Frage von Reihenfolge und Abhängigkeit zusammen \citep{schoonenboom2017}, ist aber nicht eins-zu-eins dieselbe. In jedem Fall ist vorab zu klären: Gibt es logische und/oder organisatorische Gründe, die erforderlich machen, die qualitative Komponente vor der quantitativen umzusetzen -- oder umgekehrt?
\item
  \textbf{Emergenz des Designs}: Ein Kriterium, das bei manchen Autor\emph{innen eine Schlüsselrolle spielt, bei anderen dagegen kaum beachtet wird, betrifft die Frage, wann im Forschungsprozess Designentscheidungen getroffen werden. }Emergenten Designs* stehen vorab \emph{geplante Designs} gegenüber. Dieses Kriterium verdient in meinen Augen in der Praxis durchaus, berücksichtigt zu werden. Schließlich erfordert ein emergentes Design, dass entsprechende Vorkehrungen getroffen werden. Pläne müssen flexibel gehalten werden, und um laufend Anpassungen vornehmen zu können, müssen Zeiten und Räume für Reflexion und Diskussion freigehalten werden. Emergente Designs werden häufig, aber nicht notwendigerweise, eher qualitativ ausgerichtet sein und vom Grunddesign her eher der Logik von Fallstudien folgen. Aber Teil der Idee von Emergenz ist, dass sich eine solche vorab getroffenen Einschätzungen nach und nach als irreführend herausstellen können.
\end{itemize}

Neben diesen fast zwingenden Kriterien gibt es eine Reihe von anderen Hinsichten, nach denen Mixed-Methods-Projekte typisiert werden können. So kann zum Beispiel nach den wissenschaftstheoretischen Grundlagen, nach der normativen Orientierung oder nach der (inter-)disziplinären Ausrichtung gefragt werden.

\hypertarget{wie-man-mixed-methods-typologien-sinnvoll-nutzt}{%
\section{Wie man Mixed-Methods-Typologien sinnvoll nutzt}\label{wie-man-mixed-methods-typologien-sinnvoll-nutzt}}

Die Typologien, die man in der Literatur findet, sind in der täglichen Praxis schon alleine deswegen schwierig handzuhaben, weil die meisten realen Projekte eher Misch- oder Zwischenformen sind. Das ist eigentlich nur logisch, ergibt sich doch schon bei wenigen Kriterien eine sehr hohe Zahl an theoretisch denkbaren Merkmalskombinationen (sprich: theoretisch denkbaren Typen). Die in manchen Typologien vorgesehene Restkategorie (bei Creswell \& Plano Clark wäre das wohl das ``Multiphase Design'') wird so schnell zur praktisch am häufigsten vorkommenden Variante, was kaum der Grundidee einer nützlichen Typologie entspricht.

Mixed-Methods-Typologien dürfen daher auf keinen Fall als vollständige und durch und durch systematische Taxonomien missverstanden werden. Sie sind immer Kompromisse, in die verschiedene Erwägungen einfließen (Systematik, Handhabung, Wissenschaftsverständnis \ldots). Als solche können sie hilfreich sein, weil sie Orientierung bieten und das eigene Projekt leichter darstellbar machen. Sie können aber auch kontraproduktiv sein, wenn sie zu rigide gedeutet werden und dann unflexibel und technokratisch genutzt werden. Auch Typologien sind letztlich nur Reflexionsinstrumente.

\textbf{Die Kernanregung lautet daher}: Unabhängig von der persönlich präferierten Typologie (die Auswahl ist, wie gesagt, groß) sollte zuerst und vor allem der jeweilige Kriterienkatalog durchgearbeitet werden, um das eigene Anliegen besser zu verstehen. Auf dieser Basis kann dann ein dem Projekt entsprechender ``Typ'' festgelegt werden. Diese Festlegung hilft, weil sie erlaubt, nach vergleichbaren Projekten Ausschau zu halten und die Erfahrungen anderer Forschenden für die Gestaltung des eigenen Projekts zu nutzen.

\hypertarget{notationen}{%
\chapter{Mixed Methods systematisieren II: Notationen}\label{notationen}}

\hypertarget{ein-denkbar-einfaches-system}{%
\section{Ein denkbar einfaches System \ldots{}}\label{ein-denkbar-einfaches-system}}

Neben Typologien haben sich \textbf{Notationen} als eine zweite Form durchgesetzt, Mixed-Methods-Design darzustellen. Das bis heute übliche Notationssystem stammt von Morse \citep{morse1991}. Es besteht aus wenigen Zutaten:

\begin{itemize}
\tightlist
\item
  \textbf{QUAL}: bezeichnet eine gleich- oder vorrangige qualitative Komponente
\item
  \textbf{QUAN}: bezeichnet eine gleich- oder vorrangige quantitative Komponente
\item
  \textbf{qual}: bezeichnet eine untergeorndete qualitative Komponente
\item
  \textbf{quan}: bezeichnet eine untergeordnete quantitative Komponente
\item
  \textbf{+}: bedeutet, dass Komponenten parallel (unabhängig voneinander) ablaufen
\item
  \textbf{-\textgreater{}}: bedeutet, dass eine Komponente auf die andere folgt
\end{itemize}

Mit diesem einfachen Vokabular lässt sich so gut wie jedes Mixed-Methods-Projekt abbilden:

\begin{itemize}
\tightlist
\item
  Ein ``sequenziell explanatives Design'' lässt sich mit \textbf{QUAN -\textgreater{} QUAL} beschreiben
\item
  Ein quantitativ ausgerichtetes ``embedded Design'' wäre mit \textbf{QUAN + qual} charakterisiert
\item
  Ein mehrphasiges Projekt könnte z.B. so aussehen: \textbf{(QUAN + QUAL) -\textgreater{} QUAN -\textgreater{} QUAL}
\end{itemize}

Der Operator ``-\textgreater{}'' wird in der Literatur zweideutig verwendet. Manche AutorInnen bringen mit ihm einfach eine \emph{zeitliche Reihenfolge} zum Ausdruck. Andere reservieren ihn für Fälle, in denen eine Komponente \emph{logisch} von der anderen \emph{abhängt}. Ich halte die zweite Lesart für die schlüssigere und sinnvollere, aber beide sind verbreitet.

\hypertarget{vorteile-und-schwierigkeiten-des-notationssystems}{%
\section{Vorteile und Schwierigkeiten des Notationssystems}\label{vorteile-und-schwierigkeiten-des-notationssystems}}

Die Vorteile dieses Notationssystems liegen in seiner Einfachheit und seiner Flexibilität. Diese haben natürlich ihren Preis. So wird in der Notationsschreibweise nicht sichtbar, an welchen Punkten und zu welchen Zwecken Methoden kombiniert werden -- Aspekte, die in Typologien sehr wohl zum Ausdruck gebracht werden können.

Es ist daher absolut entscheidend, dass jede Darstellung eines Mixed-Methods-Designs mittels Notation diskursiv gerahmt wird. Die Ausgestaltung jedes einzelnen Elements muss ebenso erläutert werden wie die Gründe für ihre Kombination und die exakten Punkte, an denen die Methodenintegration stattfindet (am Ende in der Phase der zusammenführenden Interpretation oder schon davor in der Datenerhebung oder während der Analyse?).

\hypertarget{die-entscheidende-erluxe4uterung-gruxfcnde-fuxfcr-die-methodenintegration}{%
\section{Die entscheidende Erläuterung: Gründe für die Methodenintegration}\label{die-entscheidende-erluxe4uterung-gruxfcnde-fuxfcr-die-methodenintegration}}

Die wichtigste rahmende Erläuterung, die jede Darstellung via Notation erfordert, betrifft die Gründe für die Kombination von qualitativen und quantitativen Methoden. In anderen Worten: jedes ``+'' und jedes ``-\textgreater{}'' verlangt nach einer verbalen Unterfütterung.
Eine hilfreiche und praxisverankerte Auflistung von Gründen für die Kombination von Methoden hat \href{https://people.utm.my/uzairiah/wp-content/blogs.dir/1541/files/2016/11/Qualitative-Research-2006-Bryman-97-113.pdf}{Bryman 2016} vorgeschlagen \citetext{\citealp{bryman2016a}; \citealp[siehe dazu auch][]{schoonenboom2017}}. Er unterscheidet beispielsweise die Motivation, einen Gegenstand aus verschiedenen Perspektiven zu untersuchen, von der Zielsetzung, eine Methode zu nutzen, um das Sampling für eine andere zu fundieren.

Die Liste von Bryman ist als erste Orientierung hilfreich. Generell zeigt sie aber wieder die zentrale Schwierigkeit jeder Systematisierung von Mixed-Methods: die Szenarien sind zu vielfältig, als das eine einfache und vollständige Schematisierung möglich wäre.

\textbf{Daher lautet die zentrale Empfehlung}: Die Gründe für die Kombination von Methoden sollten so konkret wie möglich reflektiert und dargestellt werden, immer zugschnitten auf das das jeweilige Projekt.

Qualitative Methoden können etwa eingesetzt werden, ``um zu validieren'', was quantitativ herausgefunden wurde, oder auch ``um zu erklären'', wie ein quantitatives Muster zustande kommt. Sie können auch zum Einsatz kommen, ``weil bisher wenig über einen Gegenstand bekannt ist'', der eigentlich quantitativ untersucht werden soll. Quantitative Methoden können verwendet werden, ``um die Häufigkeit zu eruieren'', mit der ein qualitativ identifiziertes Phänomen auftritt, oder ``um zu überprüfen'', ob eine qualitativ entwickelte Hypothese auch statistisch hält.

\hypertarget{visualisieren}{%
\chapter{Mixed Methods systematisieren III: Visualisieren und Integrieren}\label{visualisieren}}

Mixed Methods ist ein Prozess der ständigen Anpassung und Kohärenzsicherung

Maxwells fünf Komponenten

Denken in Arbeitspakten und Abbilden von Abläufen

Denken in inhaltlichen und methodischen Komponenten und Schnittstellen zwischen ihnen

\hypertarget{integration}{%
\chapter{Varianten der Methodenintegration}\label{integration}}

UNDER CONSTRUCTION!

\hypertarget{ohnemixing}{%
\chapter{Mixed Methods -- ohne Mixing?}\label{ohnemixing}}

UNDER CONSTRUCTION!

\#Wissenschaftstheoretische Grundlagen -- ein Prime \{\#wisstheorie\}

UNDER CONSTRUCTION!

  \bibliography{book.bib}

\end{document}
